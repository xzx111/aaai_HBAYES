Real-world organizations in business domains operate in a multi-item, multi-level environment. Items and their corresponding information collected by these organizations often reflect a hierarchical structure. For examples, products in retail stores are usually stored in hierarchical inventories. News on web pages is created and placed hierarchically in most websites. These hierarchical structures and the data within them provide a large amount of information when building effective recommendation systems. Especially in the e-commerce domain, all products are displayed in a site-wide hierarchical catalog and how to build an accurate recommendation engine on top of it becomes one of the keys to majority companies' business success. 

However, how to utilize the rich information behind hierarchical structures to make personalized and accurate product recommendations still remains challenging due to the unique characteristics of hierarchical structures and the modeling trade-offs arising from them. Briefly, most well-established recommendation algorithms cannot naturally take hierarchical structures as additional inputs and flattening decoding hierarchical structures usually doesn't work well. It will not only blow up the entire feature space but introduce noise when training the recommendation models. On the other hand, discarding hierarchies will lead to recommendation inaccuracies. The most common way to alleviate this dilemma is to feed every piece of data from the hierarchy into a complex deep neural network and hope the neural network itself can figure out a way to intelligently utilize the hierarchical knowledge. However, such approaches usually behave more like black boxes which brings much difficulty to debug and cannot provide any interpretation of the intermediate results or outcomes.

In this work, we propose and develop a hierarchical Bayesian, a.k.a., \emph{HBayes}, modeling framework that is able to flexibly capture various relations between items in hierarchical structures from different recommendation scenarios. By introducing latent variables, all hierarchical structures are encoded as conditionally independences in HBayes graphical models. Moreover, we develop a variational inference algorithm for effciently learning parameters of HBayes. 

To illustrate the power of the proposed HBayes approach, we introduce HBayes by first using a real-world apparel garment recommendation problem as an example. As an illustration, we generalize apparel styles, product brands and apparel items into a three-level hierarchy, and add additional latent variables as the apparel style membership variables to capture the diverse and hidden style properties of each brand.  Furthermore, we include user-specific features into HBayes and extend the model into the supervised learning settings where user feedback events such as clicks and conversions are incorporated.  Note that the HBayes framework is not only limited to apparel recommendation. In the end, we show its flexibility and effectiveness on another music recommendation problem as well.

Overall this paper makes contributions in four folds:

\begin{itemize}
\item It presents a generalized hierarchical Bayesian learning framework to learn from rich data with hierarchies in real cases.
\item It provides a variational inference algorithm that can learn the model parameters with very few iterations.
\item It evaluates the HBayes and its benefits comprehensively in tasks of apparel recommendation on a real-world data set. 
\item It tests the HBayes framework in different recommendation scenarios to demonstrate the model generalization and applicability.
\end{itemize}

The remainder of the paper is organized as follows: \textit{Related Work} provides a review of existing recommendation algorithms and their extensions in hierarchal learning settings.  \textit{The HBayes Framework} introduces the notations and our generalized HBayes learning framework and its variational inference algorithm. In \textit{Experiment}, we conduct experiments in a real-world e-commerce data set to show the effectiveness of our proposed recommendation algorithm in different aspects. In addition, we test our model on a music recommendation data set to illustrate the generalization and extended ability of HBayes. We summarize our work and outline potential future extensions in \textit{Conclusion} section.

