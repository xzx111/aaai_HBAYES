Traditional recommendation algorithms such as item-based approaches learn interactions between users and items and recommend items to users who share similar historical behaviors: collaborative filtering \cite{Sarwar:2001:ICF:371920.372071,Su:2009:SCF:1592474.1722966} and matrix factorization \cite{Rendle:2010:FPM} are both effective approaches under this category.  Content-based approaches including \cite{2011rsh..book...73L,Liu:2011,Yuan:2015} take use of the auxiliary information of both users and items for recommending items to users that are close in the content space.  Furthermore, session based recommender systems (RS) are developed by analyzing the session information and user visiting patterns.   \cite{Gultekin_acollaborative,Tang_review:2013} take time as an additional input for explicitly modeling user interests over time.  \cite{Koren:2010} develops a collaborative filtering approach with predictions from static average values combining with a dynamic changing factor. \cite{Yin:2011} proposes a user-tag-specific temporal interest model to track user interests over time by maximizing the time weighted data likelihood.  

Recently, there are works using Bayesian inferencing for RS.  \cite{rendle2009bpr} combines the Bayesian inference and matrix factorization together for learning users implicit feedbacks (click \& purchase) that is able to directly optimize the recommendation ranking results.   \cite{Ben-Elazar:2017,zhang2007efficient} take user preference consistency into account and develop a variational Bayesian personalized ranking model for better music recommendation.  However, these approaches do not leverage the item structural information when building their Bayesian models.  Given that the hierarchical structural information widely exists in real-world recommendation scenarios such as e-commerce, social network, music, etc. failing to utilize such information makes these Bayesian approaches inefficient and inaccurate.  

Hierarchical information is a powerful entity structure that encodes human knowledge by means of tree-based dependency constraints. RS hierarchies, in particular, could be either explicit or implicit; either approximate or exact.  There are approaches take use of such information in order to promote items to users who have explicitly visited hierarchically related items or have shown preferences to items that belong to the same sub-categories.  In social networks, \cite{shepitsen2008personalized} relies on the hierarchies generated by user-taggings to build a better personalized recommender system.  In e-commerce, \cite{wang2018exploring} introduces a hierarchical matrix factorization approach that exploits the intrinsic structural information to alleviate cold-start and data sparsity problems.  Despite the fact that these hierarchical recommender systems have received some success, there are still challenges such as: (1) how to infer the hierarchical structure efficiently and accurately if it is not explicit? (2) how to better understand the hierarchical topologies discovered by recommendation approaches? and (3) how to utilize the inferred hierarchical information for precise data explanations?

